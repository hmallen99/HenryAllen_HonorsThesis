\documentclass[../main.tex]{subfiles}

\begin{document}
A developing field in visual science is the decoding of stimuli based on brain imaging. Decoding can confirm our theories about how information is represented by the brain - we might confirm our theories about the motor system by classifying MEG signals corresponding to different muscle movements. Decoding can potentially be used to diagnose brain damage or diseases, where we might use EEG to augment ADHD diagnosis based on increased alpha frequencies. Or we might use decoding to control a brain-machine interface or decode imagery in dreams. With regards to vision, decoding has been used to confirm our theories about orientation perception and visual cortex topography. 

An interesting challenge with visual decoding is that our visual system generates a smooth, uninterrupted perception of the world from a noisy and discontinuous input stream of photons. Vision must correct for changes in lighting, muscle movements, object movements, and more, and thus visual decoding likely needs to take this correction into account. However, in addition to creating a smooth, continuous perception of the world, the visual system must also be sensitive to minute changes in the world. Adaptation is useful to the visual system because it maximizes change sensitivity. In order for the brain to react to rapid changes in the world, the visual system has to be able to quickly detect changes in orientation, colors, and object properties.Thus, the visual system has to balance between maximizing change sensitivity and creating a continuous perception of the world. The visual system can assume that the current visual input is likely to represent a physical context similar to previous visual inputs, and can introduce bias to smooth out perception. 

A perceptual bias of particular interest is visual serial dependence \citep{fischer_whitney_2014}, i.e. systemically biasing the current perception of the world towards recent inputs. The serial dependence effect is reproducible with a simple behavioral task that briefly shows a subject an oriented Gabor patch and then asks them to orient a new Gabor patch to match their perception of the stimulus. In this task, we notice that the alignment of the current stimulus is strongly attracted towards the alignment of the previous stimulant, suggesting the aforementioned systemic bias. This bias seems to last for several seconds. 

Serial dependence offers an interesting case study for understanding visual bias through visual decoding. We can easily record brain imaging data from subjects performing a serial dependence task, the effects of serial dependence are tangible and easily reproducible, and it is likely that we can account for the bias of serial dependence in a statistical model. Furthermore, previous research, like \cite{GARCIA2013515} have had success decoding oriented gratings, which are similar to the oriented Gabor patches often used in serial dependence research. This gives us good reason to think that the decoding of bias induced by serial dependence is possible.

Here we recorded MEG data from participants in an orientation judgement task. Subjects viewed a randomly oriented Gabor patch and were then asked to report the perceived orientation of the Gabor patch after the stimulus disappeared. We decoded the orientation of the stimulus Gabor patch using a decoding model trained on data from 204 MEG electrodes, and achieved accuracy above that achieved by a permutation test using the same model. We investigated the effects of serial dependence on our decoding accuracy.



\subsection{Background}
\subsubsection{Serial Dependence}
Visual input typically arrives in a discontinuous and noisy stream. The visual system has to piece together a continuous view of the world from this noisy input stream, so it uses certain bias mechanisms to smooth out our perception of the world. This paper focuses on visual serial dependence , which proposes that the present visual perception is systematically biased towards inputs from the recent past \citep{fischer_whitney_2014}. \cite{fischer_whitney_2014} first showed that a serial dependence bias existed in the perception of oriented gratings, finding that perceived orientation of the current stimulus was biased up to 10 degrees toward the previous stimulus. This bias peaked when the previous stimulus was ~20-30 degrees away from the current stimulus in either direction, decaying as the orientation difference approached 0 degrees or 90 degrees. This bias function is systemic and significant, showing that there is a scaled correction in perception toward the previous stimulus. This shows that there is a serial dependence bias at one of the basic levels of visual perception, potentially affecting primary visual cortex neurons.

Serial Dependence also occurs in larger ensembles of oriented Gabor patches \citep{Manassi}. \cite{Manassi} found that the perceived average orientation of a group of oriented Gabor patches was susceptible to the same serial dependence bias function found with single oriented gratings, though with a smaller peak bias of about 2 degrees. This ensemble serial dependence suggests that serial dependence is also associated with scene processing, i.e. calculating average orientation. Calculating average orientation, or other statistical methods performed by the visual system, are important in determining the gist of a scene; for serial dependence to appear in these integrative "gist" calculations suggests that there is a systematic bias in scene perception. This would support the theory that serial dependence is used to smooth noisy visual input streams, as the smaller steps between scene gists would result in a more continuous perception. Further, the results of \cite{Manassi} also show that serial dependence effects are relatively long-lasting, up to 5-10 seconds. 

\cite{KIYONAGA2017493} summarizes the breadth of serial dependence effects found in visual perception. Particularly, the same serial dependence effects are noticeable in face perception, suggesting that higher-level features are affected by serial dependence bias. In this study, a set of faces was created by interpolating facial features between a small set of faces. Participants were asked to morph a face towards their perception of a stimulus face shown, and it was found that this morph would be biased toward the previous face shown. \cite{KIYONAGA2017493} also noted the parallels between working memory serial dependence and visual serial dependence. These parallels highlight potential neural mechanisms behind serial dependence, such as persisting synaptic traces from previous orientations or active signals fired from higher cortical areas that induce bias. Currently, the exact neural mechanisms behind serial dependence are unknown.

\subsubsection{Imaging}
All imaging techniques suffer some weakness, whether it be invasiveness, poor temporal locality, or poor temporal locality. For example, electroencephalography (EEG) is non-invasive and has excellent temporal locality, but has poor spatial locality; fMRI is similarly non-invasive, but it has excellent spatial locality and poor temporal locality; invasive methods like single or multi-unit recording cells or eCoG have great locality, but require surgery to implement and can degrade quickly. Thus, it is best to combine imaging methods when possible, allowing for the temporal benefits of EEG and the spatial benefits of fMRI, for example. 

In this study, we primarily used magnetoencephalography, or MEG. MEG measures the weak magnetic fields that are generated by the electrical activity of neuron populations \citep{senior_russell_gazzaniga_2006}. Any electrical current will generate a perpendicular magnetic field, and neurons are no exception. The magnetic fields associated with neurons are particularly weak, as field strength falls off quadratically with distance, and the fields must penetrate through the skull. Thus, MEG primarily measures cortical activity, rather than deep brain activity, as magnetic fields associated with deep signals fall off too quickly to be measured. MEG uses incredibly powerful superconductors to pick on these magnetic fields, making the technology incredibly expensive, and very susceptible to noise. It is also clear that MEG does not react to single neuron action potentials, but instead reacts to synchronized firings of large populations of neurons. Thus, it is likely that MEG does not pick up the magnetic field associated with action potentials, which are relatively asynchronous, but the slower post-synaptic potentials associated with aligned populations of dendrites. From these aligned dendritic populations, we pick up three types of neuromagnetic fields: an anterior-posterior field, a left-right field, and a vertical field, pointing in or out from the skull.

MEG has similar properties to EEG; it is non-invasive, has good temporal locality, and poor spatial locality. MEG and EEG both suffer from the inverse problem; MEG and EEG electrodes are only able to pick up on electrical activity on the two-dimensional surface area of the scalp, and it is currently impossible to localize the origin of electromagnetic activity that could occur anywhere in the brain, especially after it passes through layers of bone and tissue. To combat this issue, structural MRI data can be used to provide information about how electrical signals would pass through the skull, allowing the signals to be localized with far greater accuracy. This process is known as source localization.

\subsubsection{Neural Decoding}
\citep{mne}

\end{document}