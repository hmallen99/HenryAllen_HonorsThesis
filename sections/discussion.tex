\documentclass[../main.tex]{subfiles}

\begin{document}
We investigated decoding orientation from MEG and measuring the effects of the previous orientation stimulus on current stimulus decoding. Our analysis particularly focused on decoding with small, continuous orientation, Gabor patches in the periphery. 


\subsection{Orientation can be decoded from MEG responses corresponding to small Gabor patch targets in the periphery}
Previous orientation decoding studies \citep{haynes_rees_2005, kamitani_tong_2005, GARCIA2013515, cichy_ramirez_pantazis_2015} have used large, foveated, oriented gratings as stimuli, and have shown that models can discriminate orientation with high accuracy from MEG and fMRI. We focused on using smaller Gabor patch targets at seven degrees eccentricity in the right visual field. This should be a harder decoding task, as there is less visual cortex area dedicated to peripheral targets that are the same physical size as foveated targets. This harder task pushes our understanding of orientation decoding further. Our orientation decoding was further complicated by the continuous range of orientations used. Many previous studies discriminated between only two orientations, usually -45 degrees and +45 degrees \citep{cichy_ramirez_pantazis_2015}, though \cite{kamitani_tong_2005} experimented with decoding eight different orientations. Our experiment was initially designed as a serial dependence study, rather than an orientation study, and used 180 different orientations, ranging evenly from 0 to 179 degrees.  We performed some initial experiments with circular regression, in which the single input orientation label was modeled in a linear regression machine by the sin and cosine of the original input. However, our initial decoding attempts were unsuccessful, and we did not pursue circular decoding. A future study might have more success fitting the MEG data to a circular regression. To solve the continuous orientation problem, we binned these orientations into nine different orientation classes for use in our multi-class decoding models.

Ultimately, the decoding of binned orientations with a variety of models was successful. This reinforces previous MEG research suggesting that orientation information can be picked up by MEG. Further, we reveal that smaller, peripheral Gabor patches can be decoded. The decoding of small orientation Gabor patches in the periphery will be instrumental in decoding more complex figures, such as natural images, from MEG. If natural images are represented by compositions of spatial frequency patches of varying size and location throughout the visual cortex, as suggested by \cite{OLSHAUSEN19973311}, then natural image decoding could perhaps be performed by correlating MEG responses to different compositions of orientation detectors. Our research expands the breadth of orientation targets that we are able to decode, making a task like natural image decoding seem more feasible. 

Nevertheless, accuracy is still fairly low in our models, and there is much work to first be done in improving MEG decoding accuracy in the periphery. Future research with small, continuous, peripheral orientations could likely benefit from a much larger training set. We had 600-800 trials for each subject, which made fitting more complex models, like a convolutional neural network or neural network with circular regression, more difficult. MEG responses can be highly sensitive to noise, and more data would could prevent overfitting in more complex models and remove some variability. Unfortunately, the creation of larger data sets is usually restricted by access to MEG and willing subjects.

\subsection{Traditional decoding models have low performance in binned multi-class orientation decoding}
Previous orientation decoding studies \citep{kamitani_tong_2005, haynes_rees_2005, cichy_ramirez_pantazis_2015} have used machine learning models like support vector machines, linear discriminant analysis, and logistic regression for classification. With the exception of \cite{kamitani_tong_2005}, which used linear discriminant analysis for multi-class decoding, many decoding studies focus on binary classification. We tested the support vector machine and logistic regression models in our multi-class decoding problem, and found that mean decoding results were significant in both cases, but with very low accuracy. Though decoding results are often barely above chance, these results were not very informative beyond the confirmation that decoding orientation was feasible. We chose MEG for its great temporal resolution, but we did not find any significant structure over time in either the SVM or logistic regression decoding. We found one time point with significance for logistic regression decoding, but this does not indicate any significant structure over time. Further, we found no time points of significance for the SVM. This could indicate that SVM and logistic regression are better suited to binary classification in decoding tasks. Particularly in our use case, where we had continuous orientations that were binned to a set of fixed orientations, there would have been much similarity when decoding orientations on the margins of these bins, resulting in noisier decoding results. Further research with our data could attempt binned decoding with linear discriminant analysis like \cite{kamitani_tong_2005}. Linear discriminant analysis could give useful predictions about how close to the margins a certain orientation might be, which could be used in temporal bias experiments.

A further reason for the low accuracy of our SVM and logistic regression models could be the circular nature of the orientation targets. Summarily, a Gabor patch oriented at 180 degrees is more similar to a Gabor patch oriented at 20 degrees than it is to a Gabor patch at 140 degrees. To mitigate this for orientations, we put the orientations at 170-180 degrees and 0-10 degrees into the same bin, but the models would have still had some difficulty decoding this circular data.


\subsection{IEM channel response has a significant bell-curve structure}
We then approached decoding with the inverted encoding model (IEM), which transforms data with a cosine basis function. We first investigated the mean channel response, and found that mean channel response had a significant bell-shaped structure, and that decoding with the mean channel response was significant. Decoding accuracy for mean channel response was about 12 percent, compared to an 11.1 percent chance accuracy. This improved upon the logistic and SVM mean accuracies, and showed that channel response drops off as a function of distance from stimulus orientation. This bell-curve structure is similar to the orientation tuning functions of V1 neurons in orientation columns, suggesting that our IEM has a similar activation structure to these V1 neurons. Further, this structure was crucial for our investigation of how previous orientations affect current orientation decoding, as we could compare minute inflections in channel response across relative previous orientations for any significance.  

\subsection{IEM channel response has significant structure over time}
While the traditional decoding models failed to show significant structure over time, we had much more success with our IEM. The IEM results showed three significant decoding time steps from 225ms to 350ms. We had also found that MEG ERFs peak from 150-200ms, suggesting that MEG decoding accuracy increases after MEG ERF peaks. Further, we found the same bell curve structure in channel response, beginning at 175ms and continuing until 350ms, lining up with our decoding results. We notice that channel response has a continuous, peaking structure here, with channel responses increasing from 175ms to a peak at 225ms, then decreasing continuously to around 325ms, before a small peak at 350ms. 

Future analysis could investigate the spatial locality of this decoding structure. In our source localized data, we found that MEG waves began moving outward from the primary visual cortex towards the visual motor areas around 150ms. A more detailed look at decoding with source localized data might be able to correlate this outward movement with IEM weights to give us a detailed look at decoding in both space and time.

Analysis could also seek to improve decoding in the pre-ERF time steps. We were unable to find any significant decoding information in the first 150ms after stimulus onset, but previous MEG studies \citep{cichy_ramirez_pantazis_2015} have found visual information encoded as soon as 50ms after stimulus onset. Our model likely suffers from more noise than the \cite{cichy_ramirez_pantazis_2015} model, as we used smaller targets with more orientations. Better decoding techniques may reveal more information at earlier time points.


\subsection{Subjects showed a serial dependence effect in behavioral responses}
Our behavioral task asked participants to orient a response bar to match the orientation of a Gabor patch they had been shown in the right visual field. We found that participant responses were significantly and systemically biased towards the orientation of the previous stimulus displayed. This bias had a maximum magnitude of about 2 degrees at 20 degrees relative previous orientation (previous - current orientation). These results are consistent with past literature in serial dependence \citep{fischer_whitney_2014, Manassi}, and suggest that the participants' current perception of the world is biased towards their previous perception of the world.

\subsection{Previous orientation can be decoded by the inverted encoding model}
Our results showed that we could significantly decode previous stimulus orientation at one time step. Interestingly, our results showed that this significant decoding occurred at 175 ms, during or after  MEG ERF peaks, but before peaks in current stimulus decoding. Because we did not find any significant decoding in the 200ms before current stimulus onset, or in the 150 ms following current stimulus onset, these results suggest that information about the previous stimulus is stored in current stimulus MEG ERFs. However, this information could be related to adaptation or serial dependence, or could be an afterimage of the previous stimulus that does not affect perception.

\subsection{Limitations of the inverted encoding model}
Although we generated statistically significant results for our current and previous orientation decoding with the inverted encoding model, we have to be careful of our interpretation of the results. 

\subsection{There is no significant relationship between previous orientation and current channel response}
We performed multiple analyses investigating any relationship between current channel response and relative previous orientation, but were ultimately unable to find a significant relationship. This does not suggest that there is no relationship between current orientation decoding and relative previous orientation, but there was no significance when particularly looking at channel response. This could result from the limitations of the inverted encoding model, which we discussed, and a circular regression approach may be able to reveal a significant structure. 

Future studies could investigate how the individual differences in serial dependence effect size relate to performance in previous orientation decoding and orientation bias in current orientation decoding. This study could establish a much stronger connection between decoding and serial dependence than we were able to make. Another interesting topic would be an analysis of subject response decoding in comparison to actual stimulus decoding. A temporal analysis here could give insight into the mechanisms behind orientation perception.





\end{document}