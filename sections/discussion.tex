\documentclass[../main.tex]{subfiles}

\begin{document}

\subsection{The orientation of small, peripheral Gabor patch targets can be decoded from MEG}
Previous orientation decoding studies \citep{haynes_rees_2005, kamitani_tong_2005, GARCIA2013515, cichy_ramirez_pantazis_2015, pantazis_fang_qin_mohsenzadeh_li_cichy_2018} have used very large gratings extending from fovea to periphery, and have shown that models can discriminate orientation with high accuracy from EEG, MEG, and fMRI. We focused on using smaller Gabor targets at seven degrees eccentricity in the right visual field. This should be a harder decoding task, as there is less visual cortex area dedicated to the periphery. This harder task pushes our understanding of orientation decoding. Our orientation decoding was further complicated by the continuous range of orientations we used. Many previous studies discriminated between only two orientations, usually -45 degrees and +45 degrees \citep{haynes_rees_2005, cichy_ramirez_pantazis_2015, GARCIA2013515}, though \cite{kamitani_tong_2005} experimented with decoding eight different orientations. \cite{pantazis_fang_qin_mohsenzadeh_li_cichy_2018} used a larger stimulus set, with 6 different orientations, each 30 degrees apart. However, Pantazis et al.'s study only performed pairwise decoding, discriminating between two out of the six stimuli at a time. 

Our experiment was designed to test serial dependence study in orientation decoding. Similar to previous studies, we used 180 different orientations, ranging evenly from 0 to 179 degrees. To solve the continuous orientation problem, we binned these orientations into nine different orientation classes for use in our multi-class decoding models \citep{haynes_rees_2005, Brouwer09, Brouwer, GARCIA2013515, cichy_ramirez_pantazis_2015}. This decoding method of binned orientations with a variety of models was successful. Our finding reinforces previous MEG research \citep{cichy_ramirez_pantazis_2015, pantazis_fang_qin_mohsenzadeh_li_cichy_2018} suggesting that orientation information can be decoded by MEG. 

Further, we reveal that smaller, peripheral Gabor patches can be decoded. Real world stimuli are much more complex than the large, oriented gratings used in previous MEG decoding studies. The smaller Gabor patch stimuli that we used are closer to modeling real-world stimuli and push the limits of MEG decoding. Future MEG studies may wish to examine the temporal dynamics of decoding natural images, for example. If natural images are represented by compositions of spatial frequency patches of varying size and location throughout the visual cortex, as suggested by \cite{OLSHAUSEN19973311}, then natural image decoding could perhaps be performed by correlating MEG responses to different compositions of orientation detectors, ranging in size, position, and spatial frequency. Our research expands the breadth of these oriented spatial frequency targets that we are able to decode, making a task like natural image decoding with MEG more feasible in the future.

Nevertheless, accuracy is low - though consistent with decoding literature - and we had to bin orientations to achieve significant decoding results. In order to improve accuracy and specificity with small, continuous, peripheral orientations, future studies would likely benefit from a larger training set. We had 600-800 trials for each subject, which has been shown to be sufficient for binary discrimination tasks. However, it is possible that decoding with small, continuous targets and MEG data requires more data in order to  prevent overfitting in more complex models and remove variability.

\subsection{Source localization decoding was not significant}
We were unable to decode stimulus orientation from source-localized activities using the logistic regression model. This could have resulted from poor model hyperparamenter tuning, as the hyperparameter tuning process was much more laborious due to the much longer training times. We may have had success using source localization with the IEM, but we decided to focus on sensor space decoding after poor initial results with logistic regression source space decoding. It's also possible that the source localization model introduced additional noise to the decoding, rather than reducing noise as intended. The main step that could have introduced noise was the co-registration process, which was where we aligned the MRI skull structure with the MEG electrodes. This involved some subjective visual estimation of how things should line up. We did verify that evoked source estimation potentials originated in the visual area, but this verification would have been biased by how we assumed the source localization results should look.

In this research, we mostly focused on the temporal results of the decoding, rather than the spatial results. With this in mind, it was not necessary to use source localization to augment our data set, as the sensor space data is sufficient for temporal investigations of decoding. The source localization process ended up using lots of computational resources and researcher time, especially for the initial cortical segmentation steps and the final model training steps. However, if future studies were able to reduce the bias introduced in the co-registration step, source localization could be incredibly useful in studies seeking to investigate decoding over time and space.

\subsection{Traditional decoding models have low performance in binned multi-class orientation decoding}
Many orientation decoding studies \citep{kamitani_tong_2005, haynes_rees_2005, cichy_ramirez_pantazis_2015} have used machine learning models such as SVMs, linear discriminant analysis (LDA), and logistic regression for classification. With the exception of \cite{kamitani_tong_2005}, which used LDA for multi-class decoding, many early decoding studies focused on binary classification. We tested the SVM and logistic regression models in our multi-class decoding problem, and found that mean decoding results were significant in both cases, but with low accuracy. Although low accuracy is typical in MEG decoding studies, these results were not informative beyond confirming that decoding the current orientation was feasible. We chose MEG for its great temporal resolution, but we did not find any significant structure over time in either the SVM or logistic regression decoding. We found one time point with significance for logistic regression decoding, but this does not indicate any significant structure over time. We found no time points of significance for the SVM. Further, we did not find any significance when decoding previous orientation from current MEG responses with either model. 

This indicates that SVM and logistic regression are not as well suited for multi-class orientation decoding as they are for two-class orientation decoding. In our use case, where we had continuous orientations that were binned to a set of fixed orientations, orientations on the margins of these bins could have introduced additional noise into our model, resulting in poorer decoding performance. Additionally, the low accuracy of our SVM and logistic regression models could be attributed to the circular nature of the orientation targets. For example, a Gabor patch oriented at 180 degrees is more similar to a Gabor patch oriented at 20 degrees than it is to a Gabor patch at 140 degrees. This is hard for an unmodified logistic regression model to encode in a loss function. Prior circular decoding research has had more success with LDA \citep{kamitani_tong_2005} and the IEM \citep{Brouwer09, Brouwer, GARCIA2013515, sprague_serences_2013, sprague_saproo_serences_2015}. LDA could give useful predictions about how close to the margins a certain orientation might be, and future studies may attempt to use LDA for orientation decoding. However, we focused on using the IEM.

\subsection{IEM channel response has a significant bell-curve structure}
We primarily approached decoding with the IEM \citep{Brouwer09, Brouwer, sprague_serences_2013, GARCIA2013515, sprague_saproo_serences_2015}, which models orientation channels with a set if cosine basis functions. We first investigated the mean channel response, and found that mean channel response had a significant bell-shaped structure, and that decoding with the mean channel response was significant. This bell-curve structure is similar to the orientation tuning functions of V1 neurons in orientation columns \citep{devalois_1978}, suggesting that our IEM has tuning similar to V1 neurons. This models orientation decoding much more naturally than logistic regression or SVMs. This improved upon the logistic regression and SVM mean accuracy, and showed that channel response drops off as a function of distance from stimulus orientation.

\subsection{IEM channel response has significant structure over time}
While the traditional decoding models failed to show significant structure over time, we had much more success showing a temporal structure with the IEM. The IEM results showed three significant decoding time steps from 225ms to 350ms. We had also found that the ERFs peak from 150 to 200ms, suggesting that MEG decoding accuracy increases after the ERF peaks. Further, we found a bell curve structure over time in channel response, beginning at 175ms and continuing until 350ms, lining up with our decoding results. We notice that channel response has a continuous, peaking structure here, with channel responses increasing from 175ms to a peak at 225ms, then decreasing continuously to around 325ms, before a small peak at 350ms. These results are mostly consistent with other MEG decoding papers \citep{cichy_ramirez_pantazis_2015, pantazis_fang_qin_mohsenzadeh_li_cichy_2018}, which found decoding accuracy to peak from around 150-300ms.

However, we were unable to find any significant decoding information in the first 150ms after stimulus onset, while a previous MEG study found visual information encoded as soon as 50ms after stimulus onset \citep{cichy_ramirez_pantazis_2015}. Our model likely suffers from more noise than the Cichy et al.'s model, as we used smaller targets with more orientations. The peak decoding accuracy time points are generally consistent across models, but the easier discrimination task in these previous studies may have been able to significantly decode orientation sooner. These studies also revealed a decoding structure over a longer period of time, up to 1000ms after stimulus onset. Our study only decoded up to 375ms after stimulus onset. We may have been able to reveal more significant information by fitting a Gaussian function to decoded channel responses and estimating the stimulus orientation as the peak orientation of the Gaussian function, as this would have played to the strengths of the IEM, rather than the all-or-none decoding accuracy scheme. Further, we may have been able to use the model outlined in \cite{vanBergen} to better model the noise of our MEG data, and revealed a more significant structure at more time points. Here, we were limited by the computational resources that this analysis would have taken.

Future analysis could also investigate the spatial locality of this decoding structure. In our source localized data, we found that MEG waves began moving from the primary visual cortex towards the anterior areas around 150ms. A more detailed look at decoding with source localized data might be able to correlate this movement of activities with IEM weights to give us a detailed look at decoding in both space and time.


\subsection{Subjects showed a serial dependence effect in behavioral responses}
Our behavioral task asked participants to orient a response bar to match the orientation of a Gabor patch they had been shown in the right visual field. We found that participant responses were significantly and systemically biased towards the orientation of the previous stimulus displayed. This bias had a maximum magnitude of about 2 degrees at 20 degrees relative previous orientation (previous - current orientation). These results are consistent with past literature in serial dependence \citep{fischer_whitney_2014, Cicchini7867, liberman_2014, Manassi, KIYONAGA2017493}, and suggest that the participants' current perception of the world is biased towards their previous perception of the world.

\subsection{Previous orientation can be decoded by the IEM}
Previous research has shown that traces of previous experiences or stimuli are found in fMRI frontal eye fields \citep{Papadimitriou} and in EEG \citep{BaeLuck}. Recent research also showed that it is possible to decode previous stimulus orientation from current stimulus-related fMRI from visual areas \citep{Sheehan2021.04.06.438664}. Our results showed that we could significantly decode previous stimulus orientation in MEG at one out of 16 time steps. Interestingly, our results showed that this significant decoding occurred at 175 ms, during or after ERF peaks, but before peaks in current stimulus decoding. Further, we show that previous orientation decoding seems to increase in the time steps before 175ms, and then decrease back to chance decoding by 225ms. This hints at a significant structure over time in previous orientation decoding. We may reveal more significant structure if we were to fit Gaussian functions to our channel responses and compare Gaussian amplitude and mean to the permutation responses. Further, higher frequency decoding analysis with more time steps could reveal a more robust structure over time.

Because we did not find any significant decoding in the 200ms before current stimulus onset, or in the 150 ms following current stimulus onset, this result suggests that information about the previous stimulus is stored in current stimulus ERFs. However, this information could be related to adaptation or serial dependence, or could be an afterimage of the previous stimulus that does not affect perception. Given that previous orientation decoding peaks just before current orientation decoding, it is possible that this previous decoding information is used to bias current orientation perception towards previous orientations, as serial dependence would suggest. We could also be decoding negative aftereffects of the previous stimulus, which might push current channel response decoding away from previous orientations.

\subsection{Results show no significant relationship between previous orientation and current channel response}
A recent fMRI study by \cite{Sheehan2021.04.06.438664} has found a repulsion effect in channel response compared to relative previous orientation, rather than the attraction effect that would be consistent with serial dependence. We performed multiple analyses investigating any relationship between current channel response and relative previous orientation, but were ultimately unable to find a significant relationship. We binned current channel responses by relative previous orientation, but found that our results were not significant by a pixel-wise permutation test of channel response. We further investigated results by fitting a Gaussian to channel responses at each time point and comparing the mean to a permutation distribution, but again found no significant results. This would suggest that channel response is not biased by relative previous orientation, though there is still much investigation to be done. Further investigation could look at time points after 375ms, when a serial dependence response may appear. Further, source localization analysis may be used to attempt decoding by specific brain regions, similar to fMRI studies. Decoding of primary visual cortex may show less of a temporal channel response bias than higher visual cortex areas, for example.

A circular regression may be more suitable for this analysis. An approach similar to \cite{Sheehan2021.04.06.438664} with MEG may also reveal significant structure with higher temporal resolution in our decoded orientations compared to relative previous orientation. This analysis uses van Bergen et al.'s modification of the IEM, and performs a circular correlation of the model bias, which we did not perform. Future studies could also investigate how the individual differences in serial dependence effect size relate to performance in previous orientation decoding and orientation bias in current orientation decoding. In our data, we found that some subjects had large serial dependence effects, up to ten degrees bias, while others showed no effect or even a mild repulsion. An individual differences study could establish a much stronger connection between decoding and serial dependence than we were able to make.

\subsection{Limitations of the IEM}
Although we found statistically significant results for our current and previous orientation decoding with the IEM, we have to be careful of our interpretation of the results. \cite{Gardner19} shows that linear transformations of the IEM basis function results in linear transformations of the decoded channel responses, essentially reconstructing some arbitrary response. This would suggest that our channel responses have a bell-curve structure because the inputs were transformed to have a bell-curve structure, and not necessarily because of the inherent shape of the channel responses. Certainly, the bell-curve transformation is a much better estimation of how similar orientations are encoded in the brain than some arbitrary linear transformation. Further, the structure of our channel responses is still significant in comparison to a permutation test, suggesting that our decoding accuracy interpretation is plausible. However, we have to be cautious when performing analysis that specifically relies on the shape of our IEM channel responses, such as the investigation of channel response bias binned by relative previous orientation. 

\cite{vanBergen} proposed a modified version of the IEM, which uses the initial transformation and weights estimation. However, it also models individual voxel (the algorithm was designed for fMRI) and total model noise. It then estimates log likelihoods for each output channel using maximum likelihood estimation. This removes the potential bias introduced by the typical IEM. Future investigations of channel response bias due to perceptual mechanisms like serial dependence would likely benefit from the use of this model, rather than the traditional IEM. If we had modeled the noise from our MEG sensors, we may have been able to find bias from the previous orientation in the current channel response.

\subsection{Conclusion}
We found that the IEM allowed us to significantly decode the orientation of small, continuous targets in the periphery. Our results suggest that current stimulus orientation decoding peaks around 225ms, after ERF peaks in occipital and temporal electrodes. These results were consistent with previous research that used large, fixated, fixed orientation targets. We were further able to decode the orientation of the previous stimulus target from current stimulus related MEG responses. We found that this decoding accuracy peaked just before current stimulus decoding. Finally, although we found a serial dependence bias effect in our behavioral data, we were unable to find any significant bias in current orientation decoding based on the relative orientation of the previous stimulus. Future studies may expand upon the methods we used here to uncover any relationship between current and previous orientation decoding.

\end{document}