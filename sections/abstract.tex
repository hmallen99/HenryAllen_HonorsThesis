\documentclass[../main.tex]{subfiles}

\begin{document}
Orientation is one of the most basic visual features encoded in the brain, and many neuroimaging studies have used decoding techniques to reveal how orientation information is represented in the brain \citep{haynes_rees_2005, kamitani_tong_2005, GARCIA2013515, cichy_ramirez_pantazis_2015}. However, few studies have investigated how orientation decoding is affected by stimulus history, despite research that shows orientation perception is biased towards recent orientation stimuli \citep{fischer_whitney_2014}. This bias towards previous stimuli - serial dependence - facilitates perceptual stability, counteracting noisy visual inputs resulting from head movements and lighting changes. In this study, we tested if current and previous orientation information can be decoded from magnetoencephalography (MEG) data. In the experiment, we showed participants small, randomly-oriented Gabor patches in the periphery, with MEG signals recorded concurrently. We decoded current and previous Gabor patch orientations at various timings relative to the stimulus onset. Our model achieved significant mean decoding performance for the current stimulus orientation, and revealed significant decoding structure from 225 to 350 ms after stimulus onset. This structure mirrors the temporal structure of event-related fields (ERFs) in the contralateral visual cortex after stimulus onset, which peaked around 150 to 225ms, on average. We also revealed a significant decoding accuracy for previous orientation at 175ms after stimulus onset. Our results suggest that both current and previous orientation information are encoded in current stimulus ERFs.

\end{document}