\documentclass[../main.tex]{subfiles}

\begin{document}
Orientation is one of the most basic visual features encoded in the brain, and many neuroimaging studies have used decoding techniques to reveal how orientation information is represented in the brain \citep{haynes_rees_2005, kamitani_tong_2005, GARCIA2013515, cichy_ramirez_pantazis_2015}. However, few studies have investigated how orientation decoding is affected by stimulus history, despite research that shows orientation perception in biased towards recent orientation stimuli \citep{fischer_whitney_2014}. Visual inputs are incredibly noisy, resulting from head movements, lighting changes, and other factors. In order to maintain a continuous perception of the world, current perception is biased towards the recent past (serial dependence). In this study, we showed 21 subjects small, randomly-oriented Gabor patches at 7 degrees eccentricity in the right visual field, with magnetoencephalography (MEG) signals recorded concurrently. We decoded current and previous Gabor orientations at various timings relative to the stimulus onset. Our model achieved significant mean decoding performance for the current stimulus orientation, and revealed significant decoding structure from 225-350 ms after stimulus onset. This structure mirrors the structure of event-related fields (ERFs) in the contralateral visual cortex after stimulus onset, which peaked around 150-225ms, on average. Our mean previous orientation decoding was not significant, but we revealed a significant decoding time point at 175ms after stimulus onset. Our results suggest that current and previous orientation information is encoded in current stimulus MEG information, and decoding accuracy peaks alongside and following ERF peaks. However, we were unable to find any significant relationship between stimulus history and bias in current orientation decoding.

\end{document}